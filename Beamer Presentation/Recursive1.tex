\begin{frame}{Recursion}
If we have two sequences

\begin{center}
    
\begin{adjustbox}{max totalsize={0.8\textwidth}{0.5\textheight}}
\begin{tikzpicture}

    \node at (0,3) {};
    \node at (2,2) {};
    \node at (2,1) {};


    \node[rectangle] at (0,2) {\Large{X}};
    \node[rectangle] at (0.5,2) {\Large{=}};
    \node[rectangle] at (1,2) {\Large{[}};
    \node[rectangle] at (1.5,2) {\Large{1}};
    \node[rectangle] at (2,2) {\Large{.}};
    \node[rectangle] at (2.5,2) {\Large{.}};
    \node[rectangle] at (3,2) {\Large{.}};
    \onslide<1,3>{\node[rectangle] at (3.5,2) {\Large{.}};}
    \onslide<2>{\node[rectangle] at (3.5,2) {\Large{a}};}
    \node[rectangle] at (4,2) {\Large{.}};
    \node[rectangle] at (4.5,2) {\Large{.}};
    \node[rectangle] at (5,2) {\Large{m}};
    \node[rectangle] at (5.5,2) {\Large{]}};

     \node[rectangle] at (7.5,2) {\Large{Y}};
    \node[rectangle] at (8,2) {\Large{=}};
    \node[rectangle] at (8.5,2) {\Large{[}};
    \node[rectangle] at (9,2) {\Large{1}};
    \node[rectangle] at (9.5,2) {\Large{.}};
    \node[rectangle] at (10,2) {\Large{.}};
    \onslide<1,3>{\node[rectangle] at (10.5,2) {\Large{.}};}
    \onslide<2>{\node[rectangle] at (10.5,2) {\Large{b}};}
    \node[rectangle] at (11,2) {\Large{.}};
    \node[rectangle] at (11.5,2) {\Large{.}};
    \node[rectangle] at (12,2) {\Large{.}};
    \node[rectangle] at (12.5,2) {\Large{n}};
    \node[rectangle] at (13,2) {\Large{]}};

    \onslide<2>\draw[very thick, red] (1.2,1.5) -- (3.8,1.5);
    \onslide<2>\draw[very thick, red] (8.8,1.5) -- (10.8,1.5);

    \onslide<3>\draw[very thick, red] (1.2,1.5) -- (5.2,1.5);
    \onslide<3>\draw[very thick, red] (8.8,1.5) -- (13.3,1.5);
   

\end{tikzpicture}
\end{adjustbox}
\end{center}

\onslide<2->{Let $LCS(a,\,b)$ denote the longest common subsequence of $x_1x_2 ... x_a$ and $y_1y_2 ... y_a$.}
\\~\
\\~\
\onslide<3->{Our goal is to find $LCS(m,\,n)$}

\end{frame}
\begin{frame}{Recursive Formulation}
\newcommand*{\xMin}{5}%
\newcommand*{\xMax}{10}%
\newcommand*{\yMin}{3}%
\newcommand*{\yMax}{11}%

\begin{tikzpicture}
    \node[rectangle] at (0,\xMin) {\large{X}};
    \node[rectangle] at (0.5,\xMin) {\large{=}};
    \node[rectangle] at (1,\xMin) {\large{[}};
    \node[rectangle] at (1.5,\xMin) {\large{1}};
    \node[rectangle] at (2,\xMin) {\large{2}};
    \node[rectangle] at (2.5,\xMin) {\large{3}};
    \node[rectangle] at (3,\xMin) {\large{.}};
    \node[rectangle] at (3.25,\xMin) {\large{.}};
    \node[rectangle] at (3.5,\xMin) {\large{.}};
    \node[rectangle] at (4,\xMin) {\large{m}};
    \node[rectangle] at (4.5,\xMin) {\large{]}};
    \onslide<2>{\draw[ultra thick, blue!60] (4-0.25,\xMin-0.25) rectangle (4+0.25,\xMin+0.25) ;}
    \onslide<2>{\draw[ultra thick, blue!60] (9.25-0.25,\xMin-0.25) rectangle (9.25+0.25,\xMin+0.25) ;}
    \onslide<3>{\draw [ultra thick, green!60] (1.25,4.75) rectangle ++(2.5,0.5);}
     \onslide<3>{\draw [ultra thick, green!60] (6.75,4.75) rectangle ++(2.25,0.5);}

    \node[rectangle] at (5.5,\xMin) {\large{Y}};
    \node[rectangle] at (6,\xMin) {\large{=}};
    \node[rectangle] at (6.5,\xMin) {\large{[}};
    \node[rectangle] at (7,\xMin) {\large{1}};
    \node[rectangle] at (7.5,\xMin) {\large{2}};
    \node[rectangle] at (8,\xMin) {\large{3}};
    \node[rectangle] at (8.25,\xMin) {\large{.}};
    \node[rectangle] at (8.5,\xMin) {\large{.}};
    \node[rectangle] at (8.75,\xMin) {\large{.}};
    \node[rectangle] at (9.25,\xMin) {\large{n}};
    \node[rectangle] at (9.75,\xMin) {\large{]}};
    \node at (0,4) {};
    
\end{tikzpicture}
    \onslide<2-3>{\large When X[m] == Y[n]}
\vspace{0.5cm}
\onslide<3>{$$ LCS(m,n) = LCS(m-1,n-1)+1 $$}
\end{frame}

\begin{frame}{Recursive Formulation}
\newcommand*{\xMin}{5}%
\newcommand*{\xMax}{10}%
\newcommand*{\yMin}{3}%
\newcommand*{\yMax}{11}%

\begin{tikzpicture}
    \node[rectangle] at (0,\xMin) {\large{X}};
    \node[rectangle] at (0.5,\xMin) {\large{=}};
    \node[rectangle] at (1,\xMin) {\large{[}};
    \node[rectangle] at (1.5,\xMin) {\large{1}};
    \node[rectangle] at (2,\xMin) {\large{2}};
    \node[rectangle] at (2.5,\xMin) {\large{3}};
    \node[rectangle] at (3,\xMin) {\large{.}};
    \node[rectangle] at (3.25,\xMin) {\large{.}};
    \node[rectangle] at (3.5,\xMin) {\large{.}};
    \node[rectangle] at (4,\xMin) {\large{m}};
    \node[rectangle] at (4.5,\xMin) {\large{]}};
    \onslide<1>{\draw[ultra thick, red!60] (4-0.25,\xMin-0.25) rectangle (4+0.25,\xMin+0.25) ;}
    \onslide<1>{\draw[ultra thick, red!60] (9.25-0.25,\xMin-0.25) rectangle (9.25+0.25,\xMin+0.25);}
    \onslide<2>{\draw [ultra thick, green!60] (1.25,4.75) rectangle ++(3,0.5);}
     \onslide<2>{\draw [ultra thick, green!60] (6.75,4.75) rectangle ++(2.25,0.5);}
     \onslide<3>{\draw [ultra thick, green!60] (1.25,4.75) rectangle ++(2.5,0.5);}
     \onslide<3>{\draw [ultra thick, green!60] (6.75,4.75) rectangle ++(2.75,0.5);}

    \node[rectangle] at (5.5,\xMin) {\large{Y}};
    \node[rectangle] at (6,\xMin) {\large{=}};
    \node[rectangle] at (6.5,\xMin) {\large{[}};
    \node[rectangle] at (7,\xMin) {\large{1}};
    \node[rectangle] at (7.5,\xMin) {\large{2}};
    \node[rectangle] at (8,\xMin) {\large{3}};
    \node[rectangle] at (8.25,\xMin) {\large{.}};
    \node[rectangle] at (8.5,\xMin) {\large{.}};
    \node[rectangle] at (8.75,\xMin) {\large{.}};
    \node[rectangle] at (9.25,\xMin) {\large{n}};
    \node[rectangle] at (9.75,\xMin) {\large{]}};
    \node at (0,4) {};
\end{tikzpicture}
    \onslide<1->{\large When X[m] != Y[n]}
\vspace{0.5cm}
\onslide<2->{$$ LCS(m,n) = max\{LCS(m,n-1),\onslide<3>{LCS(m-1,n)\} }$$}
\end{frame}

\begin{frame}{Recursive Formulation}
\newcommand*{\xMin}{5}%
\newcommand*{\xMax}{10}%
\newcommand*{\yMin}{3}%
\newcommand*{\yMax}{11}%

\begin{tikzpicture}
    \node[rectangle] at (0,\xMin) {\large{X}};
    \node[rectangle] at (0.5,\xMin) {\large{=}};
    \node[rectangle] at (1,\xMin) {\large{[}};
    \node[rectangle] at (1.5,\xMin) {\large{1}};
    \node[rectangle] at (2,\xMin) {\large{2}};
    \node[rectangle] at (2.5,\xMin) {\large{3}};
    \node[rectangle] at (3,\xMin) {\large{.}};
    \node[rectangle] at (3.25,\xMin) {\large{.}};
    \node[rectangle] at (3.5,\xMin) {\large{.}};
    \node[rectangle] at (4,\xMin) {\large{m}};
    \node[rectangle] at (4.5,\xMin) {\large{]}};


    \node[rectangle] at (5.5,\xMin) {\large{Y}};
    \node[rectangle] at (6,\xMin) {\large{=}};
    \node[rectangle] at (6.5,\xMin) {\large{[}};
    \node[rectangle] at (7,\xMin) {\large{1}};
    \node[rectangle] at (7.5,\xMin) {\large{2}};
    \node[rectangle] at (8,\xMin) {\large{3}};
    \node[rectangle] at (8.25,\xMin) {\large{.}};
    \node[rectangle] at (8.5,\xMin) {\large{.}};
    \node[rectangle] at (8.75,\xMin) {\large{.}};
    \node[rectangle] at (9.25,\xMin) {\large{n}};
    \node[rectangle] at (9.75,\xMin) {\large{]}};
    \node at (0,4) {};
\end{tikzpicture}
    \onslide<1>{\large The recursion solution:}
    \\

\small{
\[  LCS(m,n) = 
    \begin{cases} 
      LCS(m-1 , n-1)+1 & X[m] == Y[n] \\
      max\{LCS(m-1 , n),{LCS(m,n-1)\} } & X[m] != Y[n] 
   \end{cases}
\]}


\end{frame}

\begin{frame}[label=conclusion, standout]
    \begin{center}
    \LARGE What about the base case?
    \end{center}
\end{frame}

\begin{frame}{Recursive Formulation}
    \begin{center}
      \large If m == 0 then LCS(m,n) = 0\\ 
        \vspace{0.5cm}
      \large If n == 0 then LCS(m,n) = 0\\ 
    \end{center}
\end{frame}

\begin{frame}{Recursive Formulation}
    \large The final recursion solution:
       
        \small{
            \[  LCS(m,n) = 
                \begin{cases}
                  0 & m == 0 || n == 0\\
                  LCS(m-1 , n-1)+1 & X[m] == Y[n] \\
                  max\{LCS(m-1 ,n),LCS(m,n-1)\}  & X[m] != Y[n]
               \end{cases}
            \]
        }
\end{frame}



\begin{frame}[fragile]{Recursion}
\scriptsize{
$
    LCS(m,n)=\begin{cases}
    0 \quad & \text{if }\; m = 0  \quad or \quad  n = 0 \\
    LCS(m-1, n-1) + 1  \quad & \text{if }\; X[m] = Y[n] \\
    MAX(LCS(m-1,n), LCS(m,n-1)) \quad &\text{if }\; X[m] != Y[n]  \\ 
     \end{cases}
$
}
\\~\
\\~\
\small{

%\onslide<1> { \begin{tcolorbox}[colback=black!5,colframe=cyan!50!blue] } 
    \begin{minted}{python}
    def LCS(X[1..m], Y[1..n]) :
        if m = 0 or n = 0
            return 0
    \end{minted}
%\onslide<1> { \end{tcolorbox} } 
    \pause
 %\onslide<2> {     \begin{tcolorbox}[colback=black!5,colframe=green!60!blue] } 
    \begin{minted}{python}
        if X[m] = Y[n] :
            result = LCS(m-1, n-1) + 1
    \end{minted}
%\onslide<2> {\end{tcolorbox}}
    \pause
%\onslide<3> {   \begin{tcolorbox}[colback=black!5,colframe=pink!60!blue] }
    \begin{minted}{python}
        else :
            result = max(LCS(m-1, n), LCS(m, n-1))
    \end{minted}
    \pause
    \begin{minted}{python}
        return result
    \end{minted}
%\onslide<3> {\end{tcolorbox} }
}

\end{frame}



\begin{frame}{Frame Title}

\begin{center}
    

\tikzset{font=\fontsize{10}{10}\selectfont}
\begin{adjustbox}{max totalsize={\textwidth}{0.9\textheight}}
\begin{tikzpicture}[level distance=2cm,
  level 1/.style={sibling distance=8cm},
  level 2/.style={sibling distance=5cm},
  level 3/.style={sibling distance=2.5cm}]
  \node {LCS(m,n)}
     child {
      node {LCS(m-1 , n)}
      child {
        node(a0) {LCS(m-2 , n)}       
      }
      child {
        node(a1) {LCS(m-1 , n-1)}
      }
    }
     child {
      node {LCS(m , n-1)}
      child {
        node(a2) {LCS(m-1 , n-1)}
      }
      child {
        node(a3) {LCS(m-1 , n-2)}
      }
    };
\draw[ultra thick , dotted] (a0) -- +(-90:2cm);
\draw[ultra thick , dotted] (a1) -- +(-90:2cm);
\draw[ultra thick , dotted] (a2) -- +(-90:2cm);
\draw[ultra thick , dotted] (a3) -- +(-90:2cm);
\end{tikzpicture}
\end{adjustbox}
\\~\
\\~\
\onslide <2> 
{
   \textcolor{blue!80}{So the time complexity is $O(2^n)$} 
}


\end{center}
\end{frame}




\begin{frame}[fragile]{Memoized Algorithm}

\scriptsize{
   $
    LCS(m,n)=\begin{cases}
          0 \quad & \text{if }\; m = 0  \quad or \quad  n = 0 \\
           LCS(m-1, n-1) + 1  \quad & \text{if }\; X[m] = Y[n] \\
          MAX(LCS(m-1,n), LCS(m,n-1)) \quad &\text{if }\; X[m] != Y[n]  \\
     \end{cases}
$
}
\\~\
\\~\

\begin{multicols}{2}
\small{
    \begin{minted}{python}
    def LCS(X[1..m], Y[1..n]) :
        if m = 0 or n = 0
            return 0
        if LCS_memo[m][n] != -INF:
            return LCS_memo[m][n]
        else if X[m] = Y[n] :
            result = LCS(m-1, n-1) + 1
        else :
            result = max(LCS(m-1, n), LCS(m, n-1))
        LCS_memo[m][n] = result
        return result
    \end{minted}}
    
\columnbreak
\pause 
\begin{tcolorbox}[colback=green!5,colframe=green!40!black]
     \textcolor{black!80}{ Time complexity is $O(mn)$} 
\end{tcolorbox}
\end{multicols}
\end{frame}

